% Template for a Computer Science Tripos Part II project dissertation
\documentclass[12pt,a4paper,twoside,openright]{report}
\usepackage[pdfborder={0 0 0}]{hyperref}    % turns references into hyperlinks
\usepackage[margin=25mm]{geometry}  % adjusts page layout
\usepackage{graphicx}  % allows inclusion of PDF, PNG and JPG images
\usepackage{verbatim}
\usepackage{docmute}   % only needed to allow inclusion of proposal.tex

\raggedbottom                           % try to avoid widows and orphans
\sloppy
\clubpenalty1000%
\widowpenalty1000%

\renewcommand{\baselinestretch}{1.1}    % adjust line spacing to make
                                        % more readable

\begin{document}

\bibliographystyle{plain}


%%%%%%%%%%%%%%%%%%%%%%%%%%%%%%%%%%%%%%%%%%%%%%%%%%%%%%%%%%%%%%%%%%%%%%%%
% Title


\pagestyle{empty}

\rightline{\LARGE \textbf{Rupert Horlick}}

\vspace*{60mm}
\begin{center}
\Huge
\textbf{Encrypted Keyword Search Using Path ORAM on MirageOS} \\[5mm]
Computer Science Tripos -- Part II \\[5mm]
Homerton College \\[5mm]
\today  % today's date
\end{center}

%%%%%%%%%%%%%%%%%%%%%%%%%%%%%%%%%%%%%%%%%%%%%%%%%%%%%%%%%%%%%%%%%%%%%%%%%%%%%%
% Proforma, table of contents and list of figures

\pagestyle{plain}

\chapter*{Proforma}

{\large
\begin{tabular}{ll}
Name:               & \bf Rupert Horlick                       \\
College:            & \bf Homerton College                     \\
Project Title:      & \bf Encrypted Keyword Search Using \\
& \bf Path ORAM on MirageOS \\
Examination:        & \bf Computer Science Tripos -- Part II, July 2016  \\
Word Count:         & \bf 1587\footnotemark[1]
                      (well less than the 12000 limit)  \\
Project Originator: & Dr Nik Sultana                    \\
Supervisors:         & Dr Nik Sultana \& Dr Richard Mortier                    \\ 
\end{tabular}
}
\footnotetext[1]{This word count was computed
by \texttt{detex diss.tex | tr -cd '0-9A-Za-z $\tt\backslash$n' | wc -w}
}
\stepcounter{footnote}


\section*{Original Aims of the Project}

% Give a 100 word summary of what was to be achieved by the project, i.e. secure searchable encrypted documents

\section*{Work Completed}



\section*{Special Difficulties}


 
\newpage
\section*{Declaration}

I, Rupert Horlick of Homerton College, being a candidate for Part II of the Computer
Science Tripos, hereby declare
that this dissertation and the work described in it are my own work,
unaided except as may be specified below, and that the dissertation
does not contain material that has already been used to any substantial
extent for a comparable purpose.

\bigskip
\leftline{Signed [signature]}

\medskip
\leftline{Date [date]}

\tableofcontents

\listoffigures

%%%%%%%%%%%%%%%%%%%%%%%%%%%%%%%%%%%%%%%%%%%%%%%%%%%%%%%%%%%%%%%%%%%%%%%
% now for the chapters

\pagestyle{headings}

\chapter{Introduction}

% Give an overview of the problem - similar to the beginning of the project proposal

% Talk about increasing use of cloud computing

% Talk about loss of ability to search efficiently over documents

% Define threat model clearly

% Talk about solutions based on symmetric encryption

% Explain why these fail and what the consequences are

% Discuss ORAM as a solution to this problem

% Describe other work in the area of ORAM etc.

\section{Motivation}

Cloud computing is becoming ubiquitous, with more than an exabyte of data estimated to be stored in the cloud. For large businesses, a private cloud can be a cost effective way to keep data isolated, but as public cloud services become ever cheaper, even these businesses could be forced to succumb to market pressures and move to the public cloud. With so much important data held by only a few major cloud providers, trust becomes a major issue.

Encryption appears to be the solution to our trust issues; if the providers cannot read the plain-text of our data then surely it is secure? This appears to hold in general, but in the important application of query-based searching, we have a problem: using current methods of homomorphic encryption to perform search over encrypted documents can leak up to 80\% of queries \cite{islam2012access}. Knowledge about the queries made to a data set, along with the amount of documents returned by each query could lead to some dangerous inferences. As a motivating example, discovering that a query such as $\langle name,~disease\rangle$ made to a medical database returned results would allow an adversary to uncover information about a patients medical status, breaching patient confidentiality.

What allowed the authors of \cite{islam2012access} to infer search queries, was knowledge about the documents returned by any specific query. Thus, in order to protect against this kind of attack we need to have some way of preventing the server from knowing which documents it is actually returning in response to any query. Oblivious Random-Access Memory (ORAM) gives us exactly what we want. When using ORAM, not only are two accesses made to exactly the same piece of data computationally indistinguishable to the server, but so too are any two access patterns of the same length.

This project aims to demonstrate that, using a particular ORAM protocol Path ORAM \cite{stefanov2013path}, it is possible to build a system that allows us to search over a set of encrypted documents without leaking the resulting access pattern, protecting the content of the search queries and therefore the confidentiality of the documents.

\section{Challenges}

The first major challenge that faces any security related project is adequately defining the threat model. In order to be able to reason about and evaluate the security properties of the system, we need to know exactly what we assume an adversary to be capable of. Once we have done this we must show that within these capabilities, the security properties that we desire the system to have remain intact. The threat model will be defined in \S\ref{sec:threatmodel}.

By virtue of being stored in the cloud, we should be able to access our data at any time, from any place, while still maintaining the desired security properties. We also want to be tolerant of network connection errors and client-side crashes. Thus, another challenge is to make the system completely stateless.

In order to make statelessness more efficient, it is necessary to augment ORAM with recursion (\S\ref{sec:oramintro}), which reduces the amount of transient client-side storage. This enables us to efficiently store the client's state between accesses. Recursion introduces the challenge of choosing how many levels to use. Each extra level of recursion reduces the size of the client's state, but also incurs a time and space overhead. This is explored briefly in \cite{stefanov2013path}, but we will attempt to have the system automatically choose parameters for the recursion based on the size and block size of the underlying storage used by the system.

\section{Related Work}

\chapter{Preparation}

% Describe all work undertaken before the coding - show professional approach to project

% Talk about refining and clarifying model

% Talk about reviewing research papers to find best way of bringing things together

% Talk about choosing OCaml 	for its static typing and powerful module system

% Talk about how this led to designing the system in a modular way (include figure for system)

% Talk about MirageOS and building on top of it

\section{Defining the Threat Model}
\label{sec:threatmodel}

\section{Introduction to ORAM}
\label{sec:oramintro}

% Talk about Path ORAM

\section{Introduction to Inverted Indexes}

% Talk about searching using an inverted index

\section{Requirements Analysis}

% Do requirements analysis in order to show that I thought about all of these things

\section{Choice of Tools}

% Do a table of tools used including OPAM, OASIS, git, Make, Atom, etc.

\section{Software Engineering Techniques}

% Talk about writing interface files before writing code in order to fit code to requirements and designing before implementing

% Talk about the ability to do Test Driven Development and the power of unit tests

\chapter{Implementation}

This chapter describes the process that took the designs and algorithms of the previous chapter and turned them into a functioning system. As it is the core of the project, the implementation of Path ORAM is discussed first (\S\ref{sec:pathORAM}), followed by the object store (\S\ref{sec:objectStore}), the index (\S\ref{sec:index}) and finally encryption (\S\ref{sec:encryption}).

% Describe what was actually implemented

\section{Path ORAM}
\label{sec:pathORAM}

The structure of Path ORAM is described abstractly in \cite{stefanov2013path}, in terms of the core data structures

\subsection{Stash}



\subsection{Position Map}

\subsection{Basic ORAM}

\subsection{Recursion}

\subsection{Statelessness}

\subsection{Optimisation}

\begin{enumerate}
	\item Talk about optimising the block size through design space exploration
\end{enumerate}

\section{Object Store}
\label{sec:objectStore}

\subsection{General Design}

\subsection{B-Trees}

\section{Index}
\label{sec:index}

\section{Encryption}
\label{sec:encryption}

\subsection{Library}

\chapter{Evaluation}

% ### Functionality ###

% Show that the unit tests show that individual components do what they are specified to do

% Path ORAM reads in and out correctly

% Encryption does make things encrypted

% Search is sound and complete...

% Etc.

% ### Performance ###

% Evaluate the performance of the code and do it with different configurations, varying functors used, size of input, potentially simulated network latency etc.

% ### Security ###

% Use statistical methods to show that access pattern is easily visible to Bob before applying ORAM to the system and the show statistical evidence the access pattern is hidden after applying ORAM

\section{Overall Results}

\section{Unit Tests}

\section{Performance Tests}

\subsection{Microbenchmarks}

\section{Security Analysis}

\chapter{Conclusion}


%%%%%%%%%%%%%%%%%%%%%%%%%%%%%%%%%%%%%%%%%%%%%%%%%%%%%%%%%%%%%%%%%%%%%
% the bibliography
\addcontentsline{toc}{chapter}{Bibliography}
\bibliography{refs}

%%%%%%%%%%%%%%%%%%%%%%%%%%%%%%%%%%%%%%%%%%%%%%%%%%%%%%%%%%%%%%%%%%%%%
% the appendices
\appendix

\chapter{Project Proposal}

% Note: this file can be compiled on its own, but is also included by
% diss.tex (using the docmute.sty package to ignore the preamble)
\documentclass[12pt,a4paper,twoside]{article}
\usepackage[pdfborder={0 0 0}]{hyperref}
\usepackage[margin=25mm]{geometry}
\usepackage{graphicx}
\begin{document}

\vfil

\centerline{\Large Computer Science Project Proposal}
\vspace{0.4in}
\centerline{\Large Encrypted Keyword Search Using}
\vspace{0.05in}
\centerline{\Large Path ORAM on MirageOS}
\vspace{0.4in}
\centerline{\large R. Horlick, Homerton College}
\vspace{0.3in}
\centerline{\large Originator: Dr N. Sultana}
\vspace{0.3in}
\centerline{\large 23 October 2015}

\vfil


\noindent
{\bf Project Supervisors:} Dr N. Sultana \& Dr R. M. Mortier
\vspace{0.2in}

\noindent
{\bf Director of Studies:} Dr B. Roman
\vspace{0.2in}
\noindent
 
\noindent
{\bf Project Overseers:} Dr M. G. Kuhn \& Prof P. M. Sewell


% Main document

\section*{Introduction and Description of the Work}

% Set the scene - change this so that it mentions that what we want is encrypted search

As the cost of large-scale cloud storage decreases and the rate of data production grows, more and more sensitive data is being stored in the cloud. We, of course, want to encrypt our data, to ward off prying eyes, but this comes at a cost. We can no longer selectively retrieve parts of the data at will. We need some method of searching over encrypted data to find the parts we are interested in.

% Set up the threat model for "honest but curious" adversary

So let us say that Alice has a set of documents that she wants to store on an untrusted server, run by Bob. We'll first assume that Bob is ``honest, but curious'', that is, he will attempt to gather all knowledge that he can without deviating from the protocol. Alice wants to store her documents encrypted, but also wants to search over them without Bob being able to learn either the keywords she is searching for, or the results of any query, the documents that contain the keyword.

There are a number of schemes in the literature that use symmetric encryption techniques to build a searchable encryption scheme. The problem is that these all leak the access pattern, so Bob knows which documents matched any query, even if he doesn't know what they matched. It turns out that this pattern of access can leak large amounts of information. In a study [1] on an encrypted email repository, up to 80\% of search queries could be inferred from the access pattern alone! So clearly this a leak worth plugging, but how can we do it?

% Describe ORAM and mention Path ORAM

One solution to our problem is to use Oblivious Random Access Memory (ORAM), a cryptographic primitive that hides data access patterns. That is, we turn Bob's server into a block device and we attempt to maintain the property that any two sequences of accesses of the form $(operation,address,data)$, that are the same length, have computationally indistinguishable physical access patterns. Bob should have no way of learning what $address$ we are really accessing.

A trivial ORAM algorithm operates by scanning over the whole ORAM and reading/updating only the relevant block, but this has $O(N)$ bandwidth cost, where $N$ is the number of blocks, which is highly impractical for large-scale storage. Luckily, much better algorithms have been proposed. We choose to focus on Path ORAM, because it has only $O(\log N)$ bandwidth cost in the worst case if $B = \Omega(\log^2 N)$, as well as being incredibly simple conceptually.

% Extend the threat model to incorporate malicious adversary and describe integrity protection techniques

Now let's assume that Bob has become malicious, and is modifying our encrypted data. In order to combat this, we can provide integrity verification by treating the ORAM as a Merkle tree, but with data in every node. The details of this scheme are outlined below after Path ORAM has been described further.

So the project is a searchable encrypted object store, with integrity verification. It will provide a simple, name-value pair API, that allows more complex filesystems to be built on top of it. A block diagram of the system is shown in Figure \ref{miragestack}.

\begin{figure}
\setlength{\unitlength}{0.75mm}
\begin{center}
\begin{picture}(170,70)
\put(50,60){\framebox(50,10){Object Store}}

\put(100,65){\vector(1,0){20}}
\put(120,65){\vector(-1,0){20}}

\put(120,60){\framebox(50,10){Indexed Search}}

\multiput(44,55)(4,0){32}{\line(1,0){2}}
\put(0,54){BLOCK Interface}
\put(75,60){\vector(0,-1){10}}
\put(75,50){\vector(0,1){10}}

\put(145,45){\vector(0,1){15}}
\put(145,45){\vector(-1,0){45}}

\put(50,40){\framebox(50,10){ORAM}}

\multiput(44,35)(4,0){32}{\line(1,0){2}}
\put(0,34){BLOCK Interface}
\put(75,40){\vector(0,-1){10}}
\put(75,30){\vector(0,1){10}}

\put(50,20){\framebox(50,10){Encryption}}

\multiput(44,15)(4,0){32}{\line(1,0){2}}
\put(0,14){BLOCK Interface}
\put(75,20){\vector(0,-1){10}}
\put(75,10){\vector(0,1){10}}

\put(50,0){\framebox(50,10){BLOCK}}

\end{picture}
\end{center}
\caption{The Application Stack: We can use any underlying BLOCK implementation and we can add/remove ORAM, Encryption or Search modules as we please}
\label{miragestack}
\end{figure}

\section*{Starting Point}

ORAM protocols generally operate in a client-server model. For our purposes the server is the cloud storage provider and the client will be a MirageOS cloud instance. MirageOS is a unikernel operating system designed to run on the Xen hypervisor. I have chosen to use Mirage for a number of reasons. Firstly, it is lightweight and designed to be run in the cloud, meaning that simple cloud services can be built on top of it that fully leverage the ORAM. Secondly, it is written in OCaml, meaning that I can take full advantage of the rich module system. This will allow me to write my implementation as a functor that takes an implementation of Mirage's BLOCK interface and creates a new BLOCK implementation that uses ORAM, as shown in Figure \ref{miragestack}. This means that the ORAM could be used with any underlying implementation of the BLOCK interface and that it would plug seamlessly into any existing program.

% Explain Mirage (further explanation of Unikernel OS required) and justify its use

MirageOS is a framework, that pulls together a number of libraries and syntax extensions, to provide a lightweight unikernel operating system. It provides a command line tool for generating the main file, that links together implementations of various parts of the system, and passes them to the unikernel. There are a number of module signatures that define the operation of devices, such as CONSOLE for consoles, ETHIF for ethernet, and most importantly for us BLOCK, for block devices.

There are currently two implementations of the BLOCK interface, one for Unix and one for Xen, and I would like to support both. Thus I will build a functor that takes one of these and emulates a block device using it, adding ORAM to it. This will most likely use some of the algorithms/code from the existing implementations, for dealing with buffers, implemented using the Cstruct library.

% Maybe talk about nocrypto library or something

\section*{Substance and Structure of the Project}

\subsection*{Substance}

% Describe Path ORAM protocol more clearly

The Path ORAM protocol has three main components: a binary tree, a stash and a position map. The binary tree is the main storage space. Every node in the tree is a bucket, which can contain up to $Z$ blocks. The tree has height $L$, where the tree of height 0 consists only of the root node, and the leaves are at level $L$. The stash is temporary client-side storage, consisting only of a set of blocks waiting to be put back into the tree. The position map associates, with each block ID, an integer between 0 and $2^L-1$. The invariant that the Path ORAM algorithm maintains is that if the position of a block $x$ is $p$, then $x$ is either in some bucket along the path from the root node to the $p^{th}$ leaf, or in the stash. On every access to the tree, a whole path is read into, the accessed block is assigned a new random position and then as many blocks as possible are written back into the same path. This means that in any two access to the same block, the paths that are read are statistically independent.

% Describe recursive path oram protocol

We can extend the basic path ORAM algorithm with recursion. That is, calling the data ORAM $ORAM_0$, we store the position map of $ORAM_0$ in a smaller ORAM, $ORAM_1$, and the position map for this in an even smaller ORAM, $ORAM_2$. We can do this until we have a sufficiently small position map on the client. Supposing that we store $\chi$ leaf addresses in each PosMap ORAM, the position for a data block with address $a_0$ is at $a_1 = a_0 / \chi$ in $ORAM_1$, and in general $a_n = a_0 / \chi^n$ for the address in $ORAM_n$. 

% Mention that we want to be able to disconnect from the block device, so we'll need to be able to store the client state on the ORAM as well

% Talk about adding integrity verification of the ORAM

% Discuss evaluation techniques

% Move Unified ORAM to extensions

Recursion does add an overhead, but we can reduce this overhead by exploiting locality. Assuming that programs will access adjacent data blocks, we can cache PosMap blocks in a PosMap Lookaside Buffer, so that if all $\chi$ data blocks that are referenced in a PosMap block are accessed in turn we only need to do the recursion once. Doing this na\"ively, however, breaks security, because we are revealing information through the cache hit pattern. To avoid this we use Unified ORAM, which combines all of the recursive ORAMs into a single logical tree. We then use the address space to separate the levels of recursion, so addresses 1 to $N$ are for data blocks, $N + 1$ to $N + (N / \chi)$ for $ORAM_1$ and so on. Now all accesses occur in the same tree, and the security of Path ORAM keeps the cache miss pattern hidden.

A final optimisation compresses the PosMaps, reducing the number of levels of recursion required to achieve the desired client side storage, resulting in an asymptotic bandwidth complexity for ORAM with small block size.

The last two optimisations were originally designed and tested in a secure processor setting, so their application to the cloud storage setting is novel. The evaluation section of this project will provide empirical evidence as to whether these optimisations are worthwhile for our purposes.

\subsection*{Structure}

The project breaks down into the following sub-projects:

\begin{enumerate}

\item Familiarising myself with OCaml, MirageOS and related libraries

\item Implementing the basic Path ORAM functor and testing that it works in place of existing BLOCK device implementations

\item Implementing the three main optimisations to Path ORAM as extensions to the same functor, allowing for the use of different combinations of optimisations

\item Creation of a suite of tests and experiments to evaluate the relative performance of the optimisations and their conformity to theoretical bounds

\item Writing the dissertation

\end{enumerate}

\section*{Success Criterion for the Main Result}

To demonstrate, through well chosen examples, that I have implemented a secure Path ORAM functor, along with a number of ``optimisations".

\section*{Possible Extensions}

Path ORAM (and other tree based ORAMs) are limited in the fact that they have fixed-sized trees. Thus, we either need to know our storage requirements before setting up the ORAM, potentially wasting resources, or resize them in the na\"ive way as storage requirements increase. If I achieve the goals of my main project, including evaluation, ahead of schedule I will examine the possibility of making Path ORAM resizable.

\section*{Timetable: Work plan and Milestones}

Planned starting date is 16/10/2015.

\begin{enumerate}

\item {\bf 16/10/15 -- 26/10/15} Familiarise myself with relevant Mirage libraries. Implement basic Path ORAM functor.

\item {\bf 27/10/15 -- 09/11/15} Implement basic test harness. Start implementation of recursive Path ORAM.

\item {\bf 10/11/15 -- 23/11/15} 

\item {\bf 24/11/15 -- 04/12/15} 

\item {\bf 05/12/15 -- 18/12/15} 

\item {\bf 18/12/15 -- 31/12/15} 

\item {\bf 01/01/16 -- 08/01/16} 

\item {\bf 09/01/16 -- 29/01/16}

\item {\bf 30/01/16 -- 08/02/16} 

\item {\bf 09/02/16 -- 21/02/16} 

\item {\bf 22/02/16 -- 06/03/16} 

\item {\bf 07/03/16 -- 11/03/16} 

\item {\bf 12/03/16 -- 25/03/16} Final draft

\item {\bf 01/05/16 -- 13/05/16} Reread and make any final edits and then submit

\end{enumerate}

\section*{Resources Required}

\begin{itemize}

	\item My own laptop for implementation and testing

	\item My own external hard disk for backups

	\item GitHub for version control and backup storage

	\item MirageOS libraries as a basis for the project

\end{itemize}

\end{document}


\end{document}
